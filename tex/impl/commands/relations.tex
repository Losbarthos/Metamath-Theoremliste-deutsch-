% tex/impl/commands/relations.tex
\ProvidesFile{relations.tex}[2026/01/17 Relation-theoretic notation]

% ------------------------------------------------------------
% Praedikate / Begriffe zu Relationen
% (Argument wird i.d.R. als Kommaliste uebergeben, z.B. {R,A,B})
% ------------------------------------------------------------

% Totale Relation: TR(R,A,B)
\providecommand{\TotRel}[1]{\ensuremath{\mathsf{TR}(#1)}}

% (Vorbereitung fuer Ordnungen)
% Praeordnung: PreOrd(R,A)
\providecommand{\PreOrd}[1]{\ensuremath{\mathsf{PreOrd}(#1)}}

% Partielle Ordnung: PartOrd(R,A)
\providecommand{\PartOrd}[1]{\ensuremath{\mathsf{PartOrd}(#1)}}

% Totale Ordnung: TotOrd(R,A)
\providecommand{\TotOrd}[1]{\ensuremath{\mathsf{TotOrd}(#1)}}

% Strikte Ordnung: StrOrd(R,A)
\providecommand{\StrOrd}[1]{\ensuremath{\mathsf{StrOrd}(#1)}}

% (optional) Aequivalenzrelation: EqRel(R,A)
\providecommand{\EqRel}[1]{\ensuremath{\mathsf{EqRel}(#1)}}

% ------------------------------------------------------------
% commands/relations.tex (falls du die Namen als Makros willst)
% ------------------------------------------------------------
\providecommand{\Min}[1]{\ensuremath{\mathsf{Min}(#1)}}
\providecommand{\Max}[1]{\ensuremath{\mathsf{Max}(#1)}}

\DeclareMathOperator{\LB}{LB}
\DeclareMathOperator{\UB}{UB}


% ------------------------------------------------------------
% Ordnungs-Prädikate
% ------------------------------------------------------------
\DeclareRobustCommand{\WellOrd}[1]{%
  \ensuremath{\mathrm{WellOrd}\!\bigl(#1\bigr)}%
}
