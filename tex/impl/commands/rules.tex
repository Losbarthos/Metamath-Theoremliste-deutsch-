% tex/impl/commands/rules.tex
\ProvidesFile{rules.tex}[2026/01/04 ND rule macros]

% Erwartung:
% - hyperref ist bereits geladen (in der Preambel, nicht hier)
% - Labels heißen: rule:<NAME>  (z.B. rule:AI, rule:RE, ...)

% ------------------------------------------------------------
% Zentrales Helfer-Makro: erzeugt den klickbaren Regelnamen
% ------------------------------------------------------------
\providecommand{\RuleRef}[2]{%
  \hyperref[rule:#1]{\ensuremath{#2}}%
}

% ------------------------------------------------------------
% Annahme
% ------------------------------------------------------------
\providecommand{\rA}{\RuleRef{A}{A}}

% ------------------------------------------------------------
% Konjunktion ∧
% ------------------------------------------------------------
\providecommand{\rAI}[1]{\RuleRef{AI}{\land I(#1)}}
\providecommand{\rAEa}[1]{\RuleRef{AE1}{\land E1(#1)}}
\providecommand{\rAEb}[1]{\RuleRef{AE2}{\land E2(#1)}}
\providecommand{\rAEn}[1]{\RuleRef{AEn}{\land E(#1)}}

% ------------------------------------------------------------
% Disjunktion ∨
% ------------------------------------------------------------
\providecommand{\rOIa}[1]{\RuleRef{OI1}{\lor I1(#1)}}
\providecommand{\rOIb}[1]{\RuleRef{OI2}{\lor I2(#1)}}
\providecommand{\rOE}[1]{\RuleRef{OE}{\lor E(#1)}}
\providecommand{\rOEn}[1]{\RuleRef{OEn}{\lor E(#1)}}

% ------------------------------------------------------------
% Implikation →
% ------------------------------------------------------------
\providecommand{\rRI}[1]{\RuleRef{RI}{\rightarrow I(#1)}}
\providecommand{\rRE}[1]{\RuleRef{RE}{\rightarrow E(#1)}}

% ------------------------------------------------------------
% Äquivalenz ↔
% ------------------------------------------------------------
\providecommand{\rLRI}[1]{\RuleRef{LRI}{\leftrightarrow I(#1)}}
\providecommand{\rLREa}[1]{\RuleRef{LRE1}{\leftrightarrow E1(#1)}}
\providecommand{\rLREb}[1]{\RuleRef{LRE2}{\leftrightarrow E2(#1)}}
\providecommand{\rLRS}[1]{\RuleRef{LRSubst}{\leftrightarrow S(#1)}}

% ------------------------------------------------------------
% Allquantor ∀
% ------------------------------------------------------------
\providecommand{\rUI}[1]{\RuleRef{UI}{\forall I(#1)}}
\providecommand{\rUE}[1]{\RuleRef{UE}{\forall E(#1)}}

% ------------------------------------------------------------
% Existenzquantor ∃
% ------------------------------------------------------------
\providecommand{\rEI}[1]{\RuleRef{EI}{\exists I(#1)}}
\providecommand{\rEE}[1]{\RuleRef{EE}{\exists E(#1)}}

% ------------------------------------------------------------
% Eindeutige Existenz ∃!
% (Du nutzt aktuell \UEI/\UEE ohne führendes r — ich lasse das so,
% und gebe zusätzlich optionale Aliasse \rUEI/\rUEE.
% ------------------------------------------------------------
\providecommand{\UEI}[1]{\RuleRef{UEI}{\exists! I(#1)}}
\providecommand{\UEE}[1]{\RuleRef{UEE}{\exists! E(#1)}}

\providecommand{\rUEI}[1]{\UEI{#1}}
\providecommand{\rUEE}[1]{\UEE{#1}}

% ------------------------------------------------------------
% Gleichheit =
% ------------------------------------------------------------
\providecommand{\rII}{\RuleRef{II}{=I}}
\providecommand{\rIE}[1]{\RuleRef{IE}{=E(#1)}}

% Falls du wirklich separate Labels rule:rIIb / rule:rIEb hast:
\providecommand{\rIIb}[1]{\RuleRef{rIIb}{=I(#1)}}
\providecommand{\rIEb}[1]{\RuleRef{rIEb}{=E(#1)}}

\providecommand{\rNeq}{\RuleRef{Neq}{\neq}}

% ------------------------------------------------------------
% Negation ¬ und ⊥
% ------------------------------------------------------------
\providecommand{\rBI}[1]{\RuleRef{BI}{\bot I(#1)}}
\providecommand{\rCI}[1]{\RuleRef{CI}{\neg I(#1)}}
\providecommand{\rCE}[1]{\RuleRef{CE}{\neg E(#1)}}
\providecommand{\rDN}[1]{\RuleRef{DN}{DN(#1)}}

% ------------------------------------------------------------
% Kettennotation / Transitivitätsschritt
% ------------------------------------------------------------
\providecommand{\rChain}[1]{\RuleRef{Chain}{\mathsf{Tr.}(#1)}}

% ------------------------------------------------------------
% Induktion (falls du das als "Regel" klickbar willst)
% ------------------------------------------------------------
\providecommand{\rInduktion}[1]{\RuleRef{Induktion}{\mathrm{Induktion}(#1)}}
